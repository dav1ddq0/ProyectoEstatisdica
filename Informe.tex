\documentclass[a4paper,12pt]{article}
\usepackage{hyperref}
\usepackage{graphicx}
\usepackage{subcaption}
\begin{document}
\title{\huge\bf Proyecto fase 1, Estad\'istica 2020-2021}
\author{Javier E. Dom\'inguez Hern\'andez C-312\\
        David Orlando de Quesada Oliva C-311\\
        Daniel de la Cruz Prieto C-311\\        
        }
\date{21 de marzo de 2021}
\maketitle


\textbf{\large\underline{Ejercicio 1}}\\

\noindent Para trabajar este ejercicio generamos una poblacion normal con media y desviaci\'on aleatorias
(La poblacion consiste en la estatura en metros de 500 personas).\\
\begin{enumerate}
    \item[a)] Muestra 1 de la poblac\'ion (mucho mayor que 30, con reemplazo): \\\\
1.91,   1.86,   2.22,   1.44,	1.47,   1.53,
1.35,	1.76,	1.62,	1.84,   1.76,	1.81,
1.76,	1.81,	1.69,	1.55,   1.68,	1.54,
1.45,	1.86,	1.66,	1.73,   1.45,	1.91,
1.54,	1.85,	1.72,	1.33,   1.46,	1.76,
2.1,    1.69,	1.22,	1.83,   1.61,	1.53,   
1.47,	2.08,	1.77,	1.76,   2.08,	1.79,
1.48,	1.83,	1.86,	1.42,   1.59,	1.6,
1.71,	1.59,	1.32,	1.3	,   1.68,	1.58,
1.55,	1.7,    1.27,	1.66,   1.45,	1.33,
2.01,	1.75,	1.35,	1.62,   1.4,    2.2,
1.57,	1.96,	1.68,	1.86,   1.52,	1.58,
1.51,	2.22,	1.62,	1.48,   1.88,	1.99,
1.42,	1.87,	1.65,	2.12,   1.87,	1.55,
1.53,	1.36,	1.22,	1.66,   1.74,	1.52,
1.89,	1.62,	1.82,	1.54,   1.54,	1.68,
1.87,	1.67,	1.71,	1.57\\\\

\newpage
\textbf{\underline{Estad\'isticos}} \\

Medidas de tendencia central:\\\\
media = 1.6679\\
Interpretaci\'on: una persona promedio tiene una estatura 1.67m aproximadamente.\\
moda = 1.76\\
Interpretaci\'on: la mayor\'ia de las personas tiene una estatura de 1,76 aproximadamente\\
mediana = 1.66\\
Interpretaci\'on: La media de las personas tiene una estatura de 1.66 aproximadamente\\

Medidas de dispersi\'on:\\
desviaci\'on = 0.2225\\
varianza = 0.0495\\
coeficiente de variaci\'on = 0.1334\\
Interpretaci\'on : Homog\'eneo\\

Medidas de posici\'on:\\
1er cuartil = 1.5275\\
2do cuartil = 1.66\\
3er cuartil = 1.8225\\

Muestra 2 de la poblac\'ion (mayor que 30, con reemplazo): \\\\
1.72,	1.69,	1.69,	1.58,	1.69,	1.14,\\
1.29,	1.68,	1.97,	1.6	,   1.82,	1.61,\\
1.79,	1.76,	1.78,	2.33,	1.51,	1.3,\\
1.6,    1.14,	1.81,	1.54,	1.79,	1.18,\\
1.96,	1.66,	2.14,	1.77,	1.54,	1.87,\\
1.51,	1.65,	1.83,	1.79,	1.56,	1.68,\\
1.52,	1.81,	1.44,	1.71\\

\textbf{\underline{Estad\'isticos}} \\

Medidas de tendencia central:\\\\
media = 1.6613\\
Interpretaci\'on: una persona promedio tiene una estatura 1.66m aproximadamente.\\
moda = 1.69, 1.79\\
Interpretaci\'on: la mayor\'ia de las personas tiene una estatura de 1.69 o 1,79m aproximadamente\\
mediana = 1.685\\
Interpretaci\'on: La media de las personas tiene una estatura de 1.69m aproximadamente\\

Medidas de dispersi\'on:\\
desviaci\'on = 0.2434\\
varianza = 0.0593\\
coeficiente de variaci\'on = 0.1465\\
Interpretaci\'on : Homog\'eneo\\

Medidas de posici\'on:\\
1er cuartil = 1.54\\
2do cuartil = 1.685\\
3er cuartil = 1.79\\

\newpage
Muestra 3 de la poblac\'ion (20, con reemplazo): \\\\
1.69,	1.32,	1.42,	1.79,	1.6,    1.87,
1.76,	1.56,	1.18,	1.49,	1.18,	1.42,
2.06,	1.88,	1.84,	1.92,	2.08,	1.14,
1.89,	1.78\\

\textbf{\underline{Estad\'isticos}} \\

Medidas de tendencia central:\\\\
media = 1.6435\\
Interpretaci\'on: una persona promedio tiene una estatura 1.64m aproximadamente.\\
moda = 1.42, 1.18\\
Interpretaci\'on: la mayor\'ia de las personas tiene una estatura de 1.42m o 1.18 aproximadamente\\
mediana =  1.725\\
Interpretaci\'on: La media de las personas tiene una estatura de 1.73m aproximadamente\\

Medidas de dispersi\'on:\\
desviaci\'on = 0.2923\\
varianza =  0.0854\\
coeficiente de variaci\'on = 0.1779\\
Interpretaci\'on : Heterog\'eneo\\

Medidas de posici\'on:\\
1er cuartil = 1.42\\
2do cuartil = 1.725\\
3er cuartil = 1.8725\\

Muestra 4 de la poblac\'ion (30, con reemplazo): \\

1.64,	1.84,	1.4,    1.3,    1.88,	1.76,
1.18,	1.59,	2.01,	1.57,	1.36,	1.47,
1.42,	1.49,	1.33,	1.66,	1.99,	1.41,
1.61,	1.82,	1.69,	1.61,	1.75,	1.74,
1.73,	1.46,	1.45,	1.35,	1.58,	1.87\\

\textbf{\underline{Estad\'isticos}} \\

Medidas de tendencia central:\\\\
media = 1.5987\\
Interpretaci\'on: una persona promedio tiene una estatura 1.6m aproximadamente.\\
moda = 1.61\\
Interpretaci\'on: la mayor\'ia de las personas tiene una estatura de 1.61m aproximadamente\\
mediana =  1.6\\
Interpretaci\'on: La media de las personas tiene una estatura de 1.60m aproximadamente\\

Medidas de dispersi\'on:\\
desviaci\'on = 0.213\\
varianza = 0.0454\\
coeficiente de variaci\'on = 0.1332\\
Interpretaci\'on : Homog\'eneo\\

Medidas de posici\'on:\\
1er cuartil = 1.4275\\
2do cuartil = 1.6\\
3er cuartil = 1.7475\\

Muestra 5 de la poblac\'ion (mucho mayor que 30, sin reemplazo): \\\\
1.29,	1.96,	1.56,	1.76,	1.5,    1.52,
1.87,	1.91,	1.48,	1.88,	2.18,	1.58,
1.44,	1.88,	1.69,	1.91,	1.6,    1.51,
1.68,	1.49,	1.67,	1.36,	1.69,	1.84,
1.73,	1.09,	1.79,	1.67,	1.47,	1.82,
1.53,	1.49,	1.75,	1.71,	1.13,	1.36,
1.79,	1.75,	1.61,	1.43,	1.58,	1.69,
1.91,	1.73,	1.66,	2,      1.72,	1.72,
1.53,	1.39,	1.45,	1.96,	1.88,	1.52,
1.59,	1.85,	1.46,	1.66,	1.89,	1.97,
1.87,	1.62,	1.48,	1.8,    1.87,	1.74,
1.5,    1.68,	1.32,	1.92,	1.65,	1.51,
1.59,	1.59,	1.6,    1.34,	1.68,	1.38,
1.91,	1.78,	1.64,	1.48,	2.33,	1.57,
1.79,	1.61,	1.78,	2.1,    1.61,	1.7,
1.71,	2.12,	1.3,    1.54,	1.67,	1.86,
1.9,    1.83,	2.13,	1.64\\

\textbf{\underline{Estad\'isticos}}\\

Medidas de tendencia central:\\\\
media = 1.6767\\
Interpretaci\'on: una persona promedio tiene una estatura 1.68m aproximadamente.\\
moda = 1.91\\
Interpretaci\'on: la mayor\'ia de las personas tiene una estatura de 1.91m aproximadamente\\
mediana =  1.675\\
Interpretaci\'on: La media de las personas tiene una estatura de 1.68m aproximadamente\\

Medidas de dispersi\'on:\\
desviaci\'on = 0.2198\\
varianza = 0.04834\\
coeficiente de variaci\'on = 0.1311\\
Interpretaci\'on : Homog\'eneo\\

Medidas de posici\'on:\\
1er cuartil = 1.52\\
2do cuartil = 1.675\\
3er cuartil = 1.8325\\

Muestra 6 de la poblac\'ion (mayor que 30, sin reemplazo): \\

1.59,	1.59,	1.41,	1.57,	1.92,	1.64,
1.24,	1.85,	2.05,	1.35,	1.45,	1.9,
1.71,	1.64,	1.61,	1.36,	1.61,	1.62,
1.54,	1.88,	1.74,	1.64,	1.91,	1.68,
1.7,    1.69,	1.88,	2.03,	1.48,	1.44,
1.66,	2.03,	1.87,	1.76,	1.66,	1.74,
1.47,	1.52,	1.45,	2.01\\

\textbf{\underline{Estad\'isticos}}\\

Medidas de tendencia central:\\\\
media = 1.6723\\
Interpretaci\'on: una persona promedio tiene una estatura 1.67m aproximadamente.\\
moda = 1.64\\
Interpretaci\'on: la mayor\'ia de las personas tiene una estatura de 1.64m aproximadamente\\
mediana = 1.65\\
Interpretaci\'on: La media de las personas tiene una estatura de 1.65m aproximadamente\\

Medidas de dispersi\'on:\\
desviaci\'on = 0.2041\\
varianza = 0.0417\\
coeficiente de variaci\'on = 0.122\\
Interpretaci\'on : Homog\'eneo\\

Medidas de posici\'on:\\
1er cuartil = 1.535\\
2do cuartil = 1.65\\
3er cuartil = 1.855\\

Muestra 7 de la poblac\'ion (20, sin reemplazo):\\

1.53,	1.82,	1.61,	1.66,	1.74,	1.63,
1.54,	1.53,	1.69,	2.01,	2.08,	1.82,
1.64,	1.71,	1.74,	1.81,	1.5,    1.71,
1.76,	1.76\\

\textbf{\underline{Estad\'isticos}}\\

Medidas de tendencia central:\\\\
media = 1.7145\\
Interpretaci\'on: una persona promedio tiene una estatura 1.72m aproximadamente.\\
moda = 1.53, 1.82, 1.74, 1.71, 1.76\\
Interpretaci\'on: la mayor\'ia de las personas tiene una estatura de 1.64m, 1.82m, 1.74m, 1.71m, 1.76m aproximadamente\\
mediana = 1.71\\
Interpretaci\'on: La media de las personas tiene una estatura de 1.71m aproximadamente\\

Medidas de dispersi\'on:\\
desviaci\'on = 0.1503\\
varianza = 0.0226\\
coeficiente de variaci\'on = 0.0877\\
Interpretaci\'on : Muy Homog\'eneo\\

Medidas de posici\'on:\\
1er cuartil = 1.625\\
2do cuartil = 1.71\\
3er cuartil = 1.7725\\

Muestra 8 de la poblac\'ion (30, sin reemplazo): \\\\
1.66,	1.47,	1.75,	1.97,	1.67,	1.42,
1.91,	1.61,	1.78,	1.75,	1.88,	1.85,
1.58,	1.68,	1.63,	1.68,	1.67,	1.36,
1.58,	2,      1.76,	1.75,	1.74,	1.46,
1.85,	1.63,	1.45,	1.46,	1.68,	1.53\\

\textbf{\underline{Estad\'isticos}}\\

Medidas de tendencia central:\\\\
media = 1.6737\\
Interpretaci\'on: una persona promedio tiene una estatura 1.67m aproximadamente.\\
moda = 1.75, 1.68\\
Interpretaci\'on: la mayor\'ia de las personas tiene una estatura de 1.64m, 1.82m, 1.74m, 1.71m, 1.76m aproximadamente\\
mediana = 1.675\\
Interpretaci\'on: La media de las personas tiene una estatura de 1.68m aproximadamente\\

Medidas de dispersi\'on:\\
desviaci\'on = 0.1648\\
varianza = 0.0272\\
coeficiente de variaci\'on = 0.0985\\
Interpretaci\'on : Muy Homog\'eneo\\

Medidas de posici\'on:\\
1er cuartil = 1.58\\
2do cuartil = 1.675\\
3er cuartil = 1.7575\\

\item [b)] Poblaci\'on:\\\\
1.74,	1.85,	1.5,    1.51,	1.66	1.64,
1.57,	1.05,	1.21,	1.53,	1.47,	1.62,
1.66,	1.59,	1.36,	2.13,	1.74,	1.63,
1.66,	1.73,	1.48,	1.87,	1.74,	1.79,
2.12,	1.86,	1.6,    1.97,	1.58,	1.77,
1.82,	1.47,	1.62,	1.3,    1.91,	1.65,
1.75,	1.49,	1.4,    1.83,	1.68,	1.21,
1.53,	1.58,	1.47,	1.76,	1.82,	2.2,
1.9,    1.64,	1.65,	2.03,	1.51,	1.81,
1.96,	1.33,	2.2,    1.73,	1.89,	1.48,
1.71,	1.96,	1.61,	1.45,	1.7,    1.64,
1.7,    1.79,	1.36,	1.7,    1.64,	1.3,
1.47,	1.74,	1.88,	1.53,	2.02,	1.97,
1.7,    1.39,	1.91,	1.75,	2.01,	1.5,
1.52,	1.76,	1.91,	1.75,	1.49,	1.79,
1.74,	1.61,	1.71,	1.18,	1.51,	1.54,
2.06,	1.77,	1.42,	1.91,	2.18,	1.6,
1.89,	1.78,	1.94,	1.52,	2.03,	1.26,
1.67,	1.62,	1.54,	1.82,	1.66,	1.72,
1.77,	1.74,	1.79,	1.52,	1.68,	1.84,
1.72,	1.86,	1.78,	1.67,	1.24,	1.74,
1.65,	1.33,	1.67,	1.66,	1.53,	1.61,
1.58,	1.79,	2.26,	1.31,	1.91,	1.66,
1.78,	1.65,	1.68,	2,      1.57,	2.04,
1.49,	2.33,	1.65,	1.47,	1.66,	1.67,
1.81,	2,      1.83,	1.61,	1.66,	1.42,
1.45,	1.88,	1.68,	1.53,	1.6,    1.34,
1.41,	1.86,	1.77,	1.42,	1.7,    1.97,
1.37,	1.68,	1.81,	1.49,	1.91,	1.88,
1.78,	2.01,	1.45,	1.43,	1.26,	1.56,
1.76,	1.79,	1.6,    2.17,	1.46,	1.82,
1.63,	1.87,	1.5,	1.72,	1.72,	1.54,
1.6,    1.9,    2.11,	1.54,	1.44,	2.07,
1.51,	1.44,	1.14,	1.74,	1.73,	1.64,
1.38,	1.47,	1.35,	1.59,	1.92,	2.14,
1.96,	1.63,	1.85,	1.69,	1.76,	1.87,
2.08,	1.63,	1.89,	1.59,	1.68,	1.88,
1.4,    1.78,	1.54,	1.4,    1.82,	1.52,
1.69,	1.41,	1.48,	2,      1.74,	1.99,
2.06,	1.2,    1.76,	1.63,	1.85,	1.64,
1.68,	1.56,	1.71,	1.13,	1.52,	1.97,
1.96,	1.71,	1.6,    1.67,	1.83,	1.34,
2.01,	1.63,	2.03,	1.56,	1.9,    1.47,
1.83,	1.92,	1.68,	1.63,	1.87,	1.62,
1.24,	1.29,	1.99,	1.88,	1.65,	1.7,
1.55,	1.88,	2,      1.61,	1.5,    1.83,
1.68,	1.36,	1.54,	1.22,	1.37,	1.73,
1.52,	1.42,	1.54,	1.84,	1.73,	2.02,
2.08,	1.82,	1.19,	1.74,	1.83,	1.38,
1.74,	1.4,    1.7,    1.59,	1.75,	1.67,
2.22,	1.81,	1.75,	1.98,	1.78,	1.71,
2.03,	1.42,	1.47,	1.83,	1.16,	1.88,
1.84,	2.08,	1.72,	1.44,	1.79,	1.48,
1.78,	1.48,	2,      1.61,	1.85,	1.53,
2.04,	1.69,	1.72,	1.5,    1.67,	1.59,
1.61,	1.9,    1.5,    1.3,    2.1,    1.58,
1.96,	1.69,	1.76,	1.48,	1.67,	1.5,
2.05,	1.59,	1.24,	1.62,	1.55,	1.48,
1.52,	1.99,	1.78,	1.87,	1.84,	1.65,
1.64,	1.44,	1.77,	1.46,	1.51,	1.72,
1.54,	2.17,	1.59,	1.68,	2.21,	1.78,
1.37,	1.56,	1.77,	1.54,	1.73,	1.57,
1.32,	1.51,	1.36,	1.84,	1.75,	1.75,
1.96,	1.64,	1.89,	1.85,	1.66,	1.46,
1.61,	1.73,	1.58,	1.86,	1.75,	1.69,
2.03,	1.33,	1.68,	1.81,	1.54,	1.63,
1.43,	1.73,	1.74,	1.79,	1.97,	1.32,
2.29,	1.95,	1.59,	1.9,    2.16,	2.12,
1.69,	1.5,    1.84,	1.3,    1.58,	1.69,
1.45,	1.49,	1.65,	1.83,	1.49,	1.79,
1.87,	1.57,	1.9,    1.75,	1.46,	1.76,
1.81,	2.28,	0.75,	1.8,    1.72,	1.81,
1.96,	1.93,	1.69,	1.56,	1.7,    1.76,
1.58,	2.04,	1.32,	1.72,	1.85,	1.65,
2.03,	2.19,	1.46,	1.67,	1.5,    1.48,
1.39,	1.76,	1.7,    1.54,	1.48,	2.13,
1.53,	1.63,	1.72,	1.51,	1.71,	1.92,
1.86,	1.61,	1.45,	1.68,	1.89,	1.71,
2.07,	1.83,	1.5,    1.51,	1.56,	1.54,
1.61,	2.23,	1.68,	1.59,	1.92,	1.74,
1.58,	1.09,	1.54,	1.87,	1.87,	1.51,
1.27,	1.79,	1.42,	1.74,	1.75,	1.6,
1.67,	1.56,	1.5,    1.61,	1.59,	1.87,
1.64,	1.65\\

\textbf{\underline{Estad\'isticos}}\\

Medidas de tendencia central:\\\\
media = 1.6846\\
Interpretaci\'on: una persona promedio tiene una estatura 1.69m aproximadamente.\\
moda = 1.74\\
Interpretaci\'on: la mayor\'ia de las personas tiene una estatura de 1.74 aproximadamente\\
mediana = 1.68\\
Interpretaci\'on: La media de las personas tiene una estatura de 1.68m aproximadamente\\

Medidas de dispersi\'on:\\
desviaci\'on = 0.2276\\
varianza = varianza = 0.0518\\
coeficiente de variaci\'on = 0.1351\\
Interpretaci\'on : Homog\'eneo\\

Medidas de posici\'on:\\
1er cuartil = 1.53\\
2do cuartil = 1.68\\
3er cuartil = 1.83\\

\noindent Las principales diferencias de los estad\'isticos de cada muestra con respecto a los de la poblaci\'on,
se observan en las muestras de menor tama\~no, sin embargo los valores son relativamente cercanos, por lo que
se puede concluir que mientras mayor se toma la muestra mas se aproximan los estad\'isticos de la muestra a los de la poblaci\'on.\\

\item [d)] \textbf{\large\underline{Intervalos de confianza}}\\\\
Nivel de confianza asumido = 95$\%$, por lo que $\alpha$ = 0.05\\

Muestra 1 de la poblac\'ion (mucho mayor que 30, con reemplazo):\\
intervalo de confianza de la media = [1.624291 ,  1.711509]\\
intervalo de confianza de la varianza = [0.038159 ,  0.0668]\\

Muestra 2 de la poblac\'ion (mayor que 30, con reemplazo):\\ 
intervalo de confianza de la media = [1.585871 ,  1.736729]\\
intervalo de confianza de la varianza = [0.039792 ,  0.097771]\\

Muestra 3 de la poblac\'ion (20, con reemplazo): \\
intervalo de confianza de la media = [1.515396 ,  1.771604]\\
intervalo de confianza de la varianza = [0.049391 ,  0.182181]\\

Muestra 4 de la poblac\'ion (30, con reemplazo): \\
intervalo de confianza de la media = [1.52248 ,  1.67492]\\
intervalo de confianza de la varianza = [0.028796 ,  0.082046]\\

Muestra 5 de la poblac\'ion (mucho mayor que 30, sin reemplazo): \\
intervalo de confianza de la media = [1.63362 ,  1.71978]\\
intervalo de confianza de la varianza = [0.037234 ,  0.06518]\\

Muestra 6 de la poblac\'ion (mayor que 30, sin reemplazo): \\
intervalo de confianza de la media = [1.60905 ,  1.73555]\\
intervalo de confianza de la varianza = [1.60905 ,  1.73555]\\

Muestra 7 de la poblac\'ion (20, sin reemplazo):\\
intervalo de confianza de la media = [1.648629 ,  1.780371]\\
intervalo de confianza de la varianza = [0.013071 ,  0.048212]\\

Muestra 8 de la poblac\'ion (30, sin reemplazo): \\
intervalo de confianza de la media = [1.614728 ,  1.732672]\\
intervalo de confianza de la varianza = [0.017252 ,  0.049155]\\

\item [e)] Si observamos las muestras de tama\~nos similares con y sin reemplazo, debido a la forma en que
selecciona la muestra (en el primer caso la selecci\'on de un elemento de la muestra no afecta a la pr\'oxima selecci\'on,
o sea, ambos sucesos son independientes, en el segundo caso los sucesos son dependientes), algunos estad\'isticos se comportan
de forma distinta, la desviaci\'on, la varianza y el coeficiente de variaci\'onen la muestra con reemplazos es ligeramente mayor, 
ya que hay objetos repetidos en dicha muestra.\\
\end{enumerate}

\textbf{\large\underline{Ejercicio 2}}\\
\begin{enumerate}
    \item [a)] Las variables que seleccionamos de nuestro set de datos fueron accommodates : capacidad de alojamiento $($cantidad de personas$)$ en la vivienda,
price : precio de la renta, y reviews\textunderscore per\textunderscore month : cantidad de revisiones mensuales, cabe destacar que estos datos pertenecen a 1751 apartamentes disponibles
para ser rentados en la ciudad de Amberes, B\'elgica. Decidimos seleccionar estas variables, pues consideramos que son 3 aspectos que todo posible cliente quisiera
revisar antes de rentar un apartamento.\\\\

\textbf{\underline{Variable : accommodates(capacidad de alojamiento)}}\\\\
\underline{Estad\'isticos}\\\\
Medidas de tendencia central:\\\\
media = 3.6396\\
Interpretaci\'on: un apartamento promedio tiene capacidad para 4 personas aproximadamente.\\
moda = 2\\
Interpretaci\'on: la mayor\'ia de los apartamentos tiene capacidad para 2 personas aproximadamente\\
mediana = 3\\
Interpretaci\'on: La media de los apartamentos tiene capacidad para 3 personas aproximadamente\\

Medidas de dispersi\'on:\\
desviaci\'on = 2.5599\\
varianza = varianza = 6.5529\\
coeficiente de variaci\'on = 0.7033\\
Interpretaci\'on : Muy Heterog\'eneo\\

Medidas de posici\'on:\\
1er cuartil = 2\\
2do cuartil = 3\\
3er cuartil = 4\\\\

\textbf{\underline{Variable : prices(precio)}}\\\\
\underline{Estad\'isticos}\\\\
Medidas de tendencia central:\\\\
media = 94.9269\\
Interpretaci\'on: la renta de un apartamento promedio cuesta $\$$95 aproximadamente.\\
moda = 75, 60\\
Interpretaci\'on: la mayor\'ia de la renta de los apartamentos cuesta $\$$75,$\$$60\\
mediana = 71\\
Interpretaci\'on: La media de la renta de los apartamentos cuesta $\$$71\\

Medidas de dispersi\'on:\\
desviaci\'on = 122.3915\\
varianza = varianza = 14979.6747\\
coeficiente de variaci\'on = 1.2893\\
Interpretaci\'on : Muy Heterog\'eneo\\

Medidas de posici\'on:\\
1er cuartil = 50\\
2do cuartil = 71\\
3er cuartil = 100\\\\

\textbf{\underline{Variable : reviews\textunderscore per\textunderscore month(revisiones mensuales)}}\\\\
\underline{Estad\'isticos}\\\\
Medidas de tendencia central:\\\\
media = 1.2865\\
Interpretaci\'on: la cantidad de revisiones mensuales en un apartamento promedio es de 1 aproximadamente.\\
moda = 1.2865\\
Interpretaci\'on: la cantidad de revisiones mensuales de la mayor\'ia de los apartamentos es de 1 aproximadamente\\
mediana = 1.01\\
Interpretaci\'on: la media de las revisiones mensuales de los apartamentos es de 1 aproximadamente\\

Medidas de dispersi\'on:\\
desviaci\'on = 1.3002\\
varianza = varianza = 1.6905\\
coeficiente de variaci\'on = 1.0106\\
Interpretaci\'on : Muy Heterog\'eneo\\

Medidas de posici\'on:\\
1er cuartil = 0.38\\
2do cuartil = 1.01\\
3er cuartil = 1.4\\

Cabe destacar que esta \'ultima variable en el .csv que se nos entrego ten\'ia campos vac\'ios, y tuvimos que hacer ciertos
c\'omputos para rellenar dichos campos. Estuvimos investigando y decidimos rellenar dichos espacios con la media obtenida apartir
de los campos que no estaban vac\'ios, ya que como pudimos comprobar dicho valor es la mejor estimaci\'on del valor esperado, o sea el
valor que se espera para una variable aleatoria, adem\'as que de esta forma el c\'alculo de los dem\'as estad\'isticos se ve menos afectado
por la falta de informaci\'on
\newpage

    \item [b)]
    
    \item [c)]
\end{enumerate}

%\begin{center}
%   \begin{tabular}{| c | c | c |}
%    \hline
%               & proa & centro & popa \\
%    producto A & 604 & 941 & 456 \\ \hline
%    producto B & 950 & 1278 & 772 \\ \hline
%    producto C & 446 & 781 & 272 \\ \hline
%    \hline
%    \end{tabular}
%\end{center}
\end{document}